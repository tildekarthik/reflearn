\documentclass[12pt,a4paper]{article}
\usepackage[utf8]{inputenc}
\usepackage{amsmath}
\usepackage{amsfonts}
\usepackage{amssymb}
\author{Karthik A}
\title{Machine Learning -- Course summary notes and guidelines}
\begin{document}


\maketitle
\part{Algorithms}
\section{Supervised Learning}
\subsection{Regression}
Cost function:
$$J(\theta) = \frac{1}{2m}\sum_{i=1}^m(y^{(i)}-h_\theta(x^{(i)}))^2  + \frac {\lambda}{2m}\sum_{j=1}^{n}\theta_{j}^2$$

Gradient descent:
Simultaneous update: $\{$

$\theta_j := \theta_j - \frac {\alpha}{m}\frac {\partial }{\partial \theta_j}J(\theta) - \alpha \frac{\lambda}{m}\theta_j$

$\}$


Simultaneous update: $\{$

$\theta_j := \theta_j - \frac {\alpha}{m} \sum_{i=1}^{m} (h_\theta(x^{(i)})-y)x^{(i)}$

$\}$


For small $\alpha$, $J(\theta)$ should reduce with every iteration
If polynomial regression, ensure feature scaling and mean normalization

Replace each value with $(x-\mu)/s$ where $s=(x_{max}- x_{min})$ or s=Std Deviation

\subsection{Logistic Regression}

Logistic and cost function
$$h_\theta(x) = \frac{1}{(1+e^-\theta^T X)}$$
$$J(\theta) = -\frac{1}{m}\sum_{i=1}^{m}[y^{(i)}log(h_\theta(x^{(i)}))+(1-y^{(i)})log(1-h_\theta(x^{(i)}))]$$

Simultaneous update: $\{$

$\theta_j := \theta_j - \frac {\alpha}{m} \sum_{i=1}^{m} (h_\theta(x^{(i)})-y)x^{(i)}$

$\}$

\subsection{Support Vector Machines (SVM)}
\begin{itemize}
\item if n is small (10k) and m is intermedeate (50k) use SVM. If n is small and m is large -- use logistic regression. Neural network is good for all but difficult to train
\item In SVM $C=1/\lambda$
\item Without kernel or linear regression SVM is logistic regression
\end{itemize}

\subsection{Neural Network}
Back propogation - neural network terminology for minimizing our cost funtion
$\delta = a -y$, where a is the activation value\\
Layer L = Output layer\\
Layer 1 = Input layer\\
$s_j$ = units in layer j\\
$a_i^{j}$ = activation of unit i in layer j\\
$\Theta^{(j)}$ =matrix of weights controlling function mapping from layer j to layer j+1\\
$a^{j+1} = g(\Theta^TX)$\\
If $s_j$ units in layer j, $s_{j+1}$ units in layer $(j+1)$, then dimensions of $\Theta^{j}$ will be $s_{j+1}\times(s_{j}+1)$

\section{Unsupervised Learning}
\subsection{K Means and Dimensionality reduction}

How to choose K in Kmeans 
\begin{enumerate}
\item $J(\theta)$ vs K -- see if it elbows out
\item No good elbows, use the downstream purpose. For example, shirt sizes
\end{enumerate}


Dimensionality reduction
\begin{itemize}
\item Data compression
\item Visualization
\end{itemize}

\[
[U S V] = svd(\Sigma)    \;where , \Sigma  = (1/m)\sum (X*X^T)
\]

Determining number of dimensions k explaining variation of a target say 95\% 
\[Explained\;Variation = 1 - \sum_{i=1}^k s_{ii}/\sum_{i=1}^n s_{ii}\]

\subsection{Anomaly detection}
Common applications : Fraud detection, Manufacturing, monitoring machines in a large cluster


Differences between Anomaly detection and clustering
\begin{enumerate}
\item y=1 is very small , typically 0--20
\item y=0 is a large number
\item Different types of anamolies exist which are not known upfront
\end{enumerate}

Some tips for anomaly detection
\begin{itemize}
\item Add a few failed samples to CV and test, do a $F_1$ score and evaluate the $\epsilon$ values
\item When features are not gaussian distributed, use the $x^{(1/10)}$ or $x^{(1/2)}$  or $log(x)$ etc and make gaussian
\item When encountering common problem that p(x) is similar to regular in muti dimensions try,
\subitem --Introducing features such as \\$CPU Load/network traffic$ or \\$(CPU load)^2/network traffic$
\subitem --Use multivariate gaussian - automatically captures the correlation between features but is computationally expensive (also m \textgreater n else $\Sigma$ is not inversible)
\end{itemize}

\section{Recommend-er systems and COllaborative FIltering}

$n_u$ = number of users (columns)(j)
$n_m$ = number of movies (i)
$r(i,j) = 1$ if user j has rated movie i

X = hypothetical or real vector such as Genre - action, comedy each being a column $x^1$ or $x^2$ or $x^n$
$m^j$ = number of movies rated by user j

In COFI , X is also not known
\begin{itemize}
\item min $\theta$  = $\sum_i \sum _n$  and get the double gradient over X and $\theta$
\item Initialize with small random values
\item Feature scaling is not required as the rating scale is always same across uses but mean normalization helps faster convergence

\end{itemize}



\part{Practical suggestions for machine learning}
\section{Model selection and refinement}

\subsection{Over--fitting problem}
\begin{itemize}
\item Reduce the number of features
\item Use the model selection algorithm described later
\item Use regularization to reduce the magnitude of theta ($\lambda$ high)
\end{itemize}

\subsection{Model selection}
60\%--20\%--20\% is a broad split for the train, cross validation and test sets

\begin{enumerate}
\item Get  the polynomial degree -- Find $J(\theta)$ for each increasing polynomial feature. Test on CV to fix the polynomial features. Check on test 
\item If underfit reduce $\lambda$, if overfit , increase $\lambda$ \-- check the error on the CV set to fix the lambda value
\item To determine whether more or less data,  review the learning curves i.e plot the $J_{cv}$ and $J_{train}$ against increasing number of examples
\subitem ---- if $J_{cv}$ and $J_{train}$ are converging with less samples and the value is high, then it is High Bias problem and adding more data to the problem will not help
\subitem ---- if $J_{cv}$ and $J_{train}$ are different, then adding more samples will help them converge \-- high variance problem
\item Error Analysis: General guideline, start with a simple algorithm and then increase the number of features. Manually examine the errors on examples in the cross validation set and try to spot a trend where most of the errors were made. We can find some feaures that may be useful
\end{enumerate}

\subsection{Types of actions possible}
\begin{enumerate}
\item Collect more data -- High variance
\item Try smaller set of features -- High Variance
\item Try additional features -- High Bias
\item Try polynomial features -- High Bias
\item Increasing $\lambda$ -- High Variance
\item Decreasing $\lambda$ -- High Bias
\end{enumerate}



\section{Pipeline Analysis}
Example of pipeline for machine learning is given below:
Image $\rightarrow$ text detection $\rightarrow$ Character segmentation $\rightarrow$ Character recognition


\begin{itemize}
\item Sliding windows - step size or stride
\item 1D sliding window for char segmentation
\item Get lots of data and artificial data
\end{itemize}

\section{Rules for getting more data}
\begin{itemize}
\item Make sure you have a low bias classifier before starting data collection eg. add more features or hidden units in a neural network until bias is low
\item Estimate data collection effort
\item Ceiling analysis - which part of the pipeline must I improve - Look at the accuracy in the test set by feeding the $y_{actual}$ to the next part of the pipeline and see how it improves - delta accuracy should be used to determine where to put the effort
\end{itemize}


\section{Learning with large data sets}
\begin{itemize}
\item \emph{It is not who has the best algorithm that wins but who has the most data. The problem with large data sets is that gradient descent is computationally expensive for each iteration} 
\item Plot the learning curve and if high bias and J is ok, no need to do on the full data set (m) as it will converge before then -- adding more samples has no benefit
\item If high variance use map reduce -- i.e split the summation into separate machines

\end{itemize}


\subsection{Stochastic gradient} 
\begin{itemize}
\item Stochastic gradient - Do gradient descent with m=1 after shuffling the data set
\item Mini batch - do gradient descent in batches b \textless m
\item Batch gradient descent - full m
\end{itemize}

Plot J over number of iterations and average over the batch sizes
\begin{enumerate}
\item Gets smoother with smaller $\alpha$
\item Gets smoother with larger b
\item Clarifies with larger b if there is a lot of zig-zag in the chart
\item If J increases with b, use smaller $\alpha$
\item interesting but not often used is reducing $\alpha$ with iterations = $\alpha = const/(const+iterations)$

\end{enumerate}

Online learning is extreme example where the data is taken evaluated and dropped immediately

$P(y=1|r;\theta)$ - predicted CTR (Click Through Rate)



\end{document}
